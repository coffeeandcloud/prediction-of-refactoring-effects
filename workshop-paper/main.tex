\documentclass[sigconf,table,screen,xcdraw]{acmart}

\usepackage{booktabs} 
\usepackage{framed}
\usepackage{graphicx}
\usepackage{subfigure}
\usepackage{pifont}
\usepackage{xspace}
\usepackage{multirow}
\usepackage{xcolor,colortbl}
\usepackage{hhline}
\usepackage{adjustbox}
\usepackage{booktabs} 
\usepackage{amssymb}
\usepackage{balance}
\usepackage{tcolorbox}
\usepackage{url}
\usepackage{booktabs}
\usepackage{paralist}
\usepackage{enumitem}
\usepackage{lipsum}
\usepackage{listings}
\usepackage{tikz}
\usepackage{pgfplots}
\usepackage{amssymb}
\usepackage{standalone}
\usepackage{url}

\usepackage[ruled,vlined,linesnumbered]{algorithm2e}
\definecolor{codegreen}{rgb}{0,0.6,0}
\definecolor{codegray}{rgb}{0.73,0.38,0.06}
\definecolor{codepurple}{rgb}{0.27,0.38,0.97}
\definecolor{codemagenta}{rgb}{0.74,0.09,0.42}
\definecolor{backcolour}{rgb}{0.96,0.96,0.96}

% for footnotes on the same line
\usepackage[para]{footmisc}
\setcitestyle{numbers,sort&compress}

\newcommand\iris{\textsc{DisDrillery}\xspace}

\newcommand*{\ColorIfNotInString}[1]{\color{codegray}\textbf{#1}}%

\lstdefinestyle{mystyle}{
    backgroundcolor=\color{backcolour},   
    commentstyle=\color{codegreen},
    keywordstyle=\color{codepurple},
    numberstyle=\tiny\color{codegray},
    stringstyle=\color{codemagenta},
    language=Java,
    breakatwhitespace=false,         
    breaklines=true,                 
    keepspaces=true,                 
    numbers=left,
    xleftmargin=0.25cm,                    
    numbersep=5pt,                  
    showspaces=false,                
    showstringspaces=false,
    showtabs=false,                  
    tabsize=1,
    frame=tb,
    framerule=0pt,
    literate=%
    {0}{{{\ColorIfNotInString{0}}}}1
    {1}{{{\ColorIfNotInString{1}}}}1
    {2}{{{\ColorIfNotInString{2}}}}1
    {3}{{{\ColorIfNotInString{3}}}}1
    {4}{{{\ColorIfNotInString{4}}}}1
    {5}{{{\ColorIfNotInString{5}}}}1
    {6}{{{\ColorIfNotInString{6}}}}1
    {7}{{{\ColorIfNotInString{7}}}}1
    {8}{{{\ColorIfNotInString{8}}}}1
    {9}{{{\ColorIfNotInString{9}}}}1
}
% \usepackage{inconsolata}
\lstset{basicstyle=\fontsize{6}{6}\fontfamily{\ttdefault}\selectfont,style=mystyle}

\renewcommand{\arraystretch}{1.2}

\usepackage{hyperref}
\usepackage{cleveref}

\newcommand{\todo}[1]{{\color{red}TODO: #1}}
\newcommand{\citeTodo}[1]{{\color{red}[??]}}
\newcommand{\checkNum}[1]{{\color{black}$#1$}}
\newcommand{\ie}{\emph{i.e.},\xspace}
\newcommand{\Ie}{\emph{I.e.},\xspace}
\newcommand{\eg}{\emph{e.g.},\xspace}
\newcommand{\Eg}{\emph{E.g.},\xspace}
\newcommand{\etal}{\emph{et al.}\xspace}

\setlength{\floatsep}{2pt}
\setlength{\textfloatsep}{5pt}

\newcommand*\circled[1]{\tikz[baseline=(char.base)]{
    \node[shape=circle,draw,inner sep=0.5pt] (char) {#1};}}

\newenvironment{myquote}{\list{}{\leftmargin=0.08in\rightmargin=0.08in}\item[]}{\endlist}

{\newcommand{\nb}[2]{
    \fbox{\bfseries\sffamily\scriptsize#1}
    {\sf\small$\blacktriangleright$\textit{#2}$\blacktriangleleft$}
  }
\renewcommand{\arraystretch}{1.2} 
\newcommand\GIOVANNI[1]{\nb{\textcolor{red}{Gio}}{\textcolor{red}{#1}}}
\newcommand\FABIO[1]{\nb{\textcolor{red}{Fabio}}{\textcolor{red}{#1}}}
\newcommand\FABIANO[1]{\nb{\textcolor{red}{Fabiano}}{\textcolor{red}{#1}}}
\newcommand\ANDREA[1]{\nb{\textcolor{red}{Andrea}}{\textcolor{red}{#1}}}
\newcommand\HCG[1]{\nb{\textcolor{red}{Harald}}{\textcolor{red}{#1}}}

\newcommand\RQ[1]{\textbf{RQ$_#1$}}

\definecolor{gray50}{gray}{.5}
\definecolor{gray40}{gray}{.6}
\definecolor{gray30}{gray}{.7}
\definecolor{gray20}{gray}{.8}
\definecolor{gray10}{gray}{.9}
\definecolor{gray05}{gray}{.95}

\newlength\Linewidth
\def\findlength{\setlength\Linewidth\linewidth
  \addtolength\Linewidth{-4\fboxrule}
  \addtolength\Linewidth{-3\fboxsep}
}
\newenvironment{examplebox}{\par\begingroup
  \setlength{\fboxsep}{5pt}\findlength
  \setbox0=\vbox\bgroup\noindent
  \hsize=0.95\linewidth
  \begin{minipage}{0.95\linewidth}\normalsize}
  {\end{minipage}\egroup
  %    \vspace{6pt}
  \textcolor{gray20}{\fboxsep1.5pt\fbox
    {\fboxsep5pt\colorbox{gray05}{\normalcolor\box0}}}
  %    \endgroup\par\addvspace{6pt minus 3pt}\noindent
  \endgroup\par\noindent
  \normalcolor\ignorespacesafterend}
\let\Examplebox\examplebox
\let\endExamplebox\endexamplebox


\newenvironment{resultbox}{\par\begingroup
  \setlength{\fboxsep}{5pt}\findlength
  \setbox0=\vbox\bgroup\noindent
  \hsize=0.95\linewidth
  \begin{minipage}{0.95\linewidth}\normalsize}
  {\end{minipage}\egroup
  %    \vspace{6pt}
  \textcolor{gray20}{\fboxsep1.5pt\fbox
    {\fboxsep5pt\colorbox{white}{\normalcolor\box0}}}
  %    \endgroup\par\addvspace{6pt minus 3pt}\noindent
  \endgroup\par\noindent
  \normalcolor\ignorespacesafterend}
\let\Examplebox\examplebox
\let\endExamplebox\endexamplebox

%%% If you see 'ACMUNKNOWN' in the 'setcopyright' statement below,
%%% please first submit your publishing-rights agreement with ACM (follow link on submission page).
%%% Then please update our instructions page and copy-and-paste the NEW commands into your article.
%%% Please contact us in case of questions; allow up to 10 min for the system to propagate the information.
%%%
%%% The following is specific to MaLTeSQuE '20 and the paper
%%% 'Speeding Up the Data Extraction of Machine Learning Approaches: A Distributed Framework'
%%% by Martin Steinhauer and Fabio Palomba.
%%%
\setcopyright{ACMUNKNOWN}
\acmPrice{}
\acmDOI{10.1145/3416505.3423562}
\acmYear{2020}
\copyrightyear{2020}
\acmSubmissionID{fsews20maltesquemain-p5-p}
\acmISBN{978-1-4503-8124-6/20/11}
\acmConference[MaLTeSQuE '20]{Proceedings of the 4th ACM SIGSOFT International Workshop on Machine Learning Techniques for Software Quality Evaluation}{November 13, 2020}{Virtual, USA}
\acmBooktitle{Proceedings of the 4th ACM SIGSOFT International Workshop on Machine Learning Techniques for Software Quality Evaluation (MaLTeSQuE '20), November 13, 2020, Virtual, USA}


\sloppy

\begin{document}

\title[Speeding Up the Data Extraction of Machine Learning Approaches]{Speeding Up the Data Extraction of Machine Learning Approaches: A Distributed Framework}
% \iris stays for d\textbf{I}st\textbf{R}ibuted m\textbf{I}ning \textbf{S}oftware repositories

	\author{Martin Steinhauer}
	\affiliation{%
		\institution{SeSa Lab - University of Salerno}
		\city{Fisciano (Salerno)}
		\country{Italy}
	}
	\email{m.steinhauer@studenti.unisa.it}

	\author{Fabio Palomba}
	\affiliation{%
		\institution{SeSa Lab - University of Salerno}
		\city{Fisciano (Salerno)}
		\country{Italy}
	}
	\email{fpalomba@unisa.it}

\begin{abstract}
Mining software repositories (MSR) has become one of the most important sources to feed machine learning models. Nevertheless, there is still a lack of standardized pipelines to extract data in an automated and fast way. Even though several frameworks and tools exist which can fulfill specific tasks or parts of the data extraction process, none of them allow neither building an automated mining pipeline nor the possibility for full parallelization. As a consequence, researchers interested in using mining software repositories to feed machine learning models are often forced to re-implement commonly used tasks leading to additional development time and libraries may not be integrated optimally. 

This preliminary study aims to demonstrate current limitations of existing tools and Git itself which are threatening the prospects of standardization and parallelization. We also introduce the multi-dimensionality aspects of a Git repository and how they affect the computation time. Finally, as a proof of concept, we define an exemplary pipeline for predicting refactoring operations, assessing its performance. Finally, we discuss the limitations of the pipeline and further optimizations to be done.\end{abstract}

\keywords{Machine Learning Pipelines; Distributed Mining; Mining Software Repositories.}

\begin{CCSXML}
	<ccs2012>
	<concept>
	<concept_id>10011007.10011006.10011072</concept_id>
	<concept_desc>Software and its engineering~Software libraries and repositories</concept_desc>
	<concept_significance>500</concept_significance>
	</concept>
	</ccs2012>
\end{CCSXML}

\ccsdesc[500]{Software and its engineering~Software libraries and repositories}

\maketitle


\section{Introduction}
\label{sec:intro}
The success of machine learning relies heavily on the amount and quality of dataset the algorithm was trained on. This is true not only for analyzing source code repositories but all learning strategies in general. At the the same time, the generation and extraction of suitable data is often based on a manual process which is costly and time-consuming. While getting the data for other machine learning settings is often difficult, the process of mining of software repositories benefits from GitHub as one of the most important data sources and therefore has a high ability to be automated. The availability of GitHub data through big and public available open-source projects is one of the key aspects in MSR. Automation not only descreases the time spent on building datasets, it also improves the reproducibility of research projects and reusability of mining fragments throughout different projects when designed with generalization in mind.\\
However, we show in this preliminary work that the use of this potential can be extended and improved by building a pipeline for automated repository mining and also increase the computation performance by applying a distributed parallelization approach already renowned in the area of big data processing. An exemplary pipeline is built that is able to detect and extract refactoring operations directly from GitHub repositories and calculates software quality metrics before and after the refactoring operation has been conducted. Eventually,  granular performance data is collected and evaluated to affirm the potential of further improvement and work.

\section{Background and Limitations of the State of the Art}
\label{sec:background}
In this section, we provide background information on the frameworks available that ease the data extraction process of machine learning processes as well as their limitations.

\subsection{Related Work}
Previous research shows that there is a lack of reproducibility in most Git-based research projects \cite{Robles2010}. Since then, some new frameworks and libraries evolved targeting the traversal of Git projects.

Common used frameworks to analyze the commit history of software repositories are RepoDriller \cite{repodriller} and PyDriller \cite{pydriller,Spadini2018}. Whereas both tools simplify the traversing of Git commits, PyDriller is able to calculate some basic metrics like complexity and method count. The metric calculation is done within the library and could not easily be replaced with custom metric logic. Additionally, PyDriller states to be multithreaded and therefore improve the performance and lower the calculation time. A closer look into the library shows that this statement is only partially true: PyDriller is based on the library GitPython which is restricted by visiting only one commit at a time. This is caused by the underlying file-based Git archive that enables thread-safety by locking the files during e.g. check out operations. Hence it is only possible to parallelize the processing in the file dimension but not in the time dimension (the commit history). In other words, only the iteration over files within one commit and then checkout the next commit which can be parallelized. The same theory applied to RepoDriller which relies on the Java-based alternative JGit \cite{jgit}. And besides, both tools allow no multi-repository mining and require additional orchestration and process management.

Other projects try to avoid direct file system access and transferring repositories in an additional database before the first analysis. One well-known project is GHTorrent \cite{Gousios2012, Gousios2015} which allows various file-independent queries after importing the repository into a MySQL or MongoDB database. Unfortunately, the file content is not considered within this approach and only tooling for generating and importing is covered by them. 

The project `Public Git Archive' was introduced by researchers at \textsc{sourced} and enables the analysis of repositories from GitHub at a very large scale. They provide a lot of tools written in Go to query and also introduce a new storage format called Siva to reduce the stored size of forked repositories. Through a publicly available index file no additional processing is needed for developers. After the company was sold, the public available index file was taken down due to high costs and apparently was lost completely \cite{srcdissue}.

\subsection{Limitations of the State of the Art}
Despite the notable effort spent so far, in our investigation into the matter we noticed a number of limitations of the existing frameworks that can represent challenges to be addressed. 

\subsubsection{Reproducibility Challenge}
Related work shows that reproducibility of the mining pipeline is a key feature to enable other researchers an easy reproduction and verification of the used dataset. Therefore, hosted datasets can become a nightmare since they require additional (financial) resources of the project holder and also are dependent on the life cycle of the initial project. If the project gets deprecated, all data could be gone like in the case of sourced. Mining data directly from GitHub may not resolve the problem of deleted or switched to private repositories and also introduce some overhead in terms of cloning and preprocessing but decreases the dependence to a hosting company or the creator of the project. Consequently, we will focus on the raw Git archive as a source of data.

\subsubsection{Multi-Dimensionality of Git Data}
At first glance, a GitHub repository looks like a tree-like timeline which contains different revisions and file versions along the axes. But from an algorithmic perspective, that can rapidly become an issue in terms of performance as every axis results in an iteration loop to reach at least each file once. This also could be described as multi-dimensionality of the Git data, as each additional dimension added to the mining process, will result in more computation time and without considering the actual computation cost for detecting refactorings or calculating software quality metrics. Depending on the mining problem, we identified the following dimensions:

\begin{itemize}
	
    \item{\textbf{Time dimension:} Time dimension describes the total size of the analyzed commits. Git normally consists of one or more branches containing one main branch, denoted as \emph{master}. When traversing over the master branch, one will get all branches that have been merged into the master. In our case, we assume that for refactoring detection it would be sufficient to detect all valuable refactorings that have been considered as useful by merging them back into the master branch. Therefore, the time dimension of the refactoring detection could be seen as linear. When an analysis problem requires visiting every commit in the repository, the time dimension is considered as the total count of commits within the whole repository. }
    
    \smallskip
    \item{\textbf{File dimension:} Defined by the file count one commit has in total. This amount of files can vary significantly and often depend on the project size, age of the project or the used programming language. Additionally, it is important to note that the file dimension is different for each commit as a file can be added or deleted within a commit. If the latest commit has only a few files and all others have many, the file dimension will affect the performance more as if the last commit have many files and the others have only a few. }
    
    \smallskip
    \item{\textbf{Structural node dimension:} Similar to the file dimension, the total number of interesting nodes, for example, classes, methods or code lines, are counted in the content of each file. Considering object-oriented languages, a file normally contains one class, but also could consist of more than one class declaration. If the mining problem is dependent on method or class member declarations, those would also be considered as a number of interesting nodes per file.}
    
    \smallskip
    \item{\textbf{Metric dimension:} The last dimension is very specific particularly regarding the mining problem. In our example, we would like to extract refactoring operations and calculate software quality metrics. Therefore, the metric dimension is highly coupled with the structural node dimension and defined by the method the metric is collected. RefactoringMiner \cite{Tsantalis:ICSE:2018:RefactoringMiner}} uses an abstract UML algorithm to detect refactorings, while CK \cite{aniche-ck} is using an abstract syntax tree (AST). Those algorithms also add on top of the file or structural dimension, depending on how they are implemented and designed.
\end{itemize}

\subsubsection{Parallelization Challenge}
One of the biggest challenges is building a mining tool that allows scaling of the mining process along the presented dimensionality axes. Both presented tools, PyDriller and RepoDriller, do not or only partially support the parallelization. While RepoDriller does not do any kind of parallelization, PyDriller allows scaling along the file dimension axis. Files within a single commit can be analyzed by multiple threads and increase the performance and decrease the computation time as the use of those libraries shows \cite{Gote2019}. As the multi-dimensionality of GitHub repositories outlines, we can not presume that repositories contain many files. When a repository contains fewer files but has a large revision history, the runtime of the mining process will get worse.

As already explained, the structural node dimension and the metric dimension are highly dependent on the type of analysis that should be executed. To make this approach generalizable, we focus on parallelization along the time and file dimension axes. As seen in PyDriller, the file dimension can easily be parallelized. In opposite, the commit axis is rather difficult: Since Git writes lock files when accessing the repository to avoid file inconsistencies by modifying the repository, only one commit can be visited concurrently. This limitation reflects in JGit and GitPython, which has all file-based actions marked as non-threadsafe. This limitation results in libraries not implementing a time dimension parallelization approach.

\subsubsection{Generalization Challenge}
Even though we are conducting this distributed approach in the context of detecting refactoring operations, it is designed to be generalizable and can be reused for other types of mining problems. Since mining problems can vary from social aspects up to high technical questions like code evolution and smell detection, it is not easy to build a pipeline that matches all type of problems. Therefore, this pipeline concentrates on technical, content-based repository mining. Additionally, libraries are not standardized in their input and output design. While RefactoringMiner takes the repository and commits as an input, CK is not designed Git-specific and takes just the source directory as an input. This requires the pipeline to adopt the corresponding libraries in terms of inputs and outputs of each pipeline step and even more important to keep track of the memory usage, since the libraries itself write outputs to disk, while within a pipeline, results partially are stored in memory. This necessitates careful pipeline design and memory management to avoid out of memory exceptions.

\section{\iris: A Distributed Framework to Support Machine Learning Research}
\label{sec:approach}

TBD

\section{Preliminary Assessment of \iris}
\label{sec:assessment}

\subsection{Experimental Settings}
\label{sec:method}
The \emph{goal} of the study is to provide a preliminary assessment of the performance of the proposed framework, with the \emph{purpose} of understanding its potential usefulness for the data extraction stage of machine learning approaches. More specifically, we seek to understand computational time ... Hence, we pose the following research question:

\begin{center}
	\begin{examplebox}
		\textbf{RQ$_1$.} \emph{To what extent can \iris improve the computational time with respect to a single-threaded baseline?}	
	\end{examplebox}
\end{center}

\textbf{Context selection.} The \emph{context} of the study is composed of the version history of the systems reported in Table \ref{tab:systems}. The selection process is driven by... 


\subsection{Analysis of the results}
\label{sec:results}



\section{Discussion and Limitations}
\label{sec:discussion}

\subsection{Discussion}
\label{sec:observations}
Here we should put some discussion/observations about the quantitative results. Why is the approach not able to reach even lower computational time? What about the overhead? etc. 

\subsection{Limitations of the framework}
\label{sec:limitations}


\section{Conclusion and Future Work}
\label{conclusion}
The presented prototype of the \iris framework clearly shows that parallelization of the time dimension can increase the performance in MSR. We also discussed the major issues that affecting the runtime negatively and analyzed the current problems of the implementation. Therefore, there is a lot of potential improving the prototype in terms of parallelization as well as generalization. The following section is presenting further steps to optimize the framework and make it more flexible for other mining tasks.\\
\textbf{Replacing the file-based repository by database.}
This step is needed to avoid problems introduced by file-based Git repositories and the described locking mechanism. This approach is inspired by the GHTorrent project \cite{Gousios2012, Gousios2015} and the Sourced project \cite{Markovtsev2018}. Unlike GHTorrent, we'd like to additionally store file contents together with meta-information about the project in a large database. To avoid hosting problems like in Sourced and improve reproducibility, our pipeline should support the automatic generation of this database by a given list of interesting repositories. It should be evaluated, which storage format is most suitable in interoperability with Spark (e.g. Apache Cassandra \cite{lakshman2010cassandra}, Neo4J \cite{neo4j} or just simple parquet files \cite{parquet}).\\
Using a database as input not only avoids the file lock in Git repositories, it is also likely to improve the overall performance since Spark can use query optimization and higher data distribution since it is not limited to file-based data sources. Also, basic operations like counting commits along a branch can be done much faster in a structured and pre-processed environment. The pre-processing overhead has to be evaluated.\\
\textbf{Generalization through standardized inputs and outputs.} Building a pipeline upon Apache Spark improves a standardized design of each analysis step. Building well-defined input and output formats for each processing step raise the level of reusability of certain processing steps and allow further research much easier adoption of other research questions.\\
\textbf{Graphs by default.} Spark enhances big data analysis by integrated support for graph data and adjusted, for distributed environments optimized graph algorithms \cite{graphx}. If the dataset provides access to graph-like structures makes it easier answering research-questions in the area of developer-interaction.\\
Additionally, it could be considered to store source code information not as plain text but as graph data using abstract syntax trees (AST). Smaller projects have already constructed such use cases in combination with Neo4j \cite{Arora2019}.\\
\textbf{Reproducibility by default.} Through pre-processing, the data becomes much easier to deal with and besides, there is no dependency on existing datasets because it can be generated by everyone just by accessing the public Git API. Furthermore, Spark allows us to run on common cloud platforms like Amazon AWS, Azure, or Google Cloud as well as on local clusters or even on a developer's laptop. This makes executing the pipeline fairly easy. 

\balance
\bibliographystyle{abbrv}
\bibliography{references}
% That's all folks!
\end{document}
